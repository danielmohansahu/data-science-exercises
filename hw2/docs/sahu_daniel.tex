% Adapted from
%  https://www.overleaf.com/latex/templates/pitt-state-physics-homework-template/wdsxknmntnxk

\documentclass[12pt]{article}
\usepackage[paper=letterpaper,margin=2cm]{geometry}
\usepackage{amsmath}
\usepackage{amssymb}
\usepackage{amsfonts}
\usepackage{newtxtext, newtxmath}
\usepackage{enumitem}
\usepackage{titling}
\usepackage{graphicx}
\usepackage{enumitem}
\usepackage[colorlinks=true]{hyperref}

\setlength{\droptitle}{-6em}

% Enter the specific assignment number and topic of that assignment below, and replace "Your Name" with your actual name.
\title{ENPM808W Homework \#2}
\author{Daniel M. Sahu}
\date{\today}

\begin{document}
\maketitle
\begin{enumerate}[leftmargin=\labelsep]
\item \textbf{Downloaded housing data from ELMS.}

  N/A

\item \textbf{Predicting 2013 house cost via Linear Regression with State information.}

  Linear Regression is generally straightforward, but in our case there is a pre-process required because State information is categorical - there's no way to interpret it sequentially. To handle this we convert our single variable (State) into N different boolean variables (e.g. CA vs. Not CA, DC vs. Not DC). One interpretation of this is that we're fitting N different lines to each subset of data corresponding to a single state.

  \begin{enumerate}[label=(\alph*)]
  \item What is the intercept? What does it correspond to?

    The intercept is 12.563. In a line (\(y = mx + b\)) this is the \(b\), or the point at which this intercepts the Y axis. In our case we're using categorical information so the Y axis interpretation is less useful. Instead we must combine this with the coefficient information for a given state to interpret it properly.

  \item How do you get this information from your regression?

    This information is embedded in the regression model itself.

  \item Based on your regression coefficients, what states have the most and least expensive average homes?

    Based on regression coefficients the most expensive "state" is the District of Columbia (Washington DC) with a regression coefficient of (0.45). The cheapest state is West Virginia with a coefficient of (-1.20). We can use the coefficient itself as a proxy for home cost, because the predictive model for a home in the given state is given by:

      \[ cost = e^{coefficient + intercept} \]

    Note that the \(x\) in \(y = mx + b\) is 1.0.

  \item How do you get this information from your regression?

    The coefficient information, like the intercept, is embedded directly in the trained regression model. In order to find each state specific coefficient we simply find the corresponding index of the dummy state variable (i.e. the feature "DC or Not DC").

  \item What is the average price of homes in those states?

    Using the formula in section (c) we get:

    The average price of homes in DC is:

      \[ e^{coefficient + intercept} = e^{0.45 + 12.563} \approx \$450,900 \]

    The average price of homes in WV is:

      \[ e^{coefficient + intercept} = e^{-1.2 + 12.563} \approx \$85,733 \]

  \item How do you get this information from your regression?

    This information is directly extracted via knowledge of the linear regression model type (linear) and the pre-processing step we used to avoid overfitting to expensive homes (the natural logarithm).

  \end{enumerate}

\item \textbf{Predicting 2013 house cost via Linear Regression with State and County information.}

  \begin{enumerate}[label=(\alph*)]
  \item What US counties have the highest and lowest regression coefficients? Why?

    The U.S. county with the lowest regression coefficient is Meriwether, and the highest is Anchorage. This is slightly more complicated to explain than the purely State-based regression model developed in Question \#2, because the total cost of a given location is contingent on both the State and County weights. In this case the most intuitive representation is that Anchorage represents a very expensive county such that the county weight must, well, out-weight the state regression coefficient. Similarly it's possible that Meriwether represents a very cheap county in a very rich state, as opposed to a globally minimally expensive County. The presence of multiple variables significantly complicates the interpretation of the data.

  \end{enumerate}

\item \textbf{Kaggle Housing Contest submission.}

  My submission model is a linear regressor using 3 features:
  \begin{enumerate}
    \item poverty (provided)
    \item price2007 (provided, logarithm taken)
    \item weighted cost based on location
  \end{enumerate}

  The last feature is the main innovation of this submission. It is a weighted sum of the log of the distance from the zip code of the city to all other cities (calculated via the python pgeocode library), with the weight being the 2007 home price of each other city. The purpose of this feature is to encode continuous location information into the regression model. 

  Contest information:
  \begin{enumerate}
    \item Username: dsahu\_117007190
    \item Score: 100140.46757
  \end{enumerate}

\item \textbf{Probability Exercise: Ball Selection}

  Given:
    \[ P(W | B1) = 2/3 \]
    \[ P(W | B2) = 3/4 \]
    \[ P(B1) = P(B2) = 1/2 \]

  Where \(P(W)\) is the probability of selecting a White ball, \(P(B1)\) is the probability of selecting Bag \#1, and \(P(B2)\) is the probability of selecting Bag \#2.
  Since the full suite of possibilities is covered by \(P(B1)\) and \(P(B2)\) (i.e. \( P(B1) + P(B2) = 0.5 \)), we can calculate the probability of selecting a White ball as:

    \[ P(W) = P(W | B1) P(B1) + P(W | B2) P(B2) \]
    \[ P(W) = 2/3 * 1/2 + 3/4 * 1/2 = 17/24 \]

  The chance of randomly selecting a White ball is \(P(W) = 17/24\).

\item \textbf{Probability Exercise: Soccer Scores}

  Given:
    \[ P(W | G) = 0.6 \]
    \[ P(W | NG) = 0.1 \]
    \[ P(G) = 0.3 \]

  Where \(P(W)\) is the probability of Winning, \(P(G)\) is the probability of getting the first goal, and \(P(NG)\) is the probability of not getting the first goal.
  
  As stated this problem is underconstrained and cannot be answered. This is because there is no information given about the prevalence of tied games. We make the simplifying assumption that tie games \emph{never} occur, which makes the problem tractable. This effectively gives us the following constraining equation:

    \[ P(NG) = 0.7 \]

  Now that the full suite of possible cases is known, we can formulate the probability of winning as:

    \[ P(W) = P(W | G) P(G) + P(W | NG) P(NG) \]
    \[ P(W) = 0.6 * 0.3 + 0.1 * 0.7 = 0.25 \]

    The chance of this team winning any given game is roughly 25\%. We say "roughly" because the problem statement says "the team scores the first goal \emph{about} 30\% of the time" (emphasis mine).

\end{enumerate}
\end{document}
